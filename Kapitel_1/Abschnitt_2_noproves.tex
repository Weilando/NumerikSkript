\subsection{Ordnung von Quadraturformeln}

\begin{definition}
Eine Quadraturformel (QF) mit Gewichten und Knoten $(b_i, c_i)_{i=1}^{s}$ hat \textbf{Ordnung p}, falls sie exakt ist für alle Polynome von Grad $ \leq p-1$.\\
$\mathcal{P}$: Menge aller Polynome $$\left\{ \sum_{i=0}^{n}a_i*X^i , a_i \in \mathbb{R} (\mathbb{C}) \right\}  $$ \\
deg(q): Grad des Polynoms
\end{definition}

\begin{theorem}
Ein QF $(b_i, c_i)_{i=1}^{s}$ für $[0,1]$ hat Ordnung $p$ genau dann, wenn
$$\sum_{i=1}^{s} b_i c_i^{q-1} = \frac{1}{q}$$ für $q = 1,..,p$
\end{theorem}

\begin{example}
\begin{description}
  \item
\end{description}

\begin{enumerate}
  \item Rechtecksregel: $p=1$
  \item Mittelpunktsregel: $p=2$
  \item Trapezregel: $p=2$
  \item Simpsonregel: $p \geq 3$ nach Konstruktion \\
  $q = 4$: $1/6 * 0^3 + 4/6 * (1/2)^3 + 1/6 * 1^3 = 1/4 = 1/4$ \\
  $q = 5$: $1/6 * 0^4 + 4/6 * (1/2)^4 + 1/6 * 1^4 = 5/24 \neq 1/5$ \\
  Damit ist die Ordnung 4!
  \item "pulcherina et utilissima": Übung
\end{enumerate}
\end{example}

\begin{comment}
Zu vergebenen paarweise verschiedenen Knoten $c_1, ..., c_s$ lässt sich aus (*) für $p=s$ ein lineares Gleichungssystem für die Gewichte $b_1, ..., b_s$ aufstellen.\\

$$
\underbrace{\left[ \begin{array}{rrrr}
1 & 1 & ... & 1 \\
c_1 & c_2 & ... & c_s \\
... & ... & ... & ... \\
c_1^{s-1} & c_2^{s-1} & ... & c_s^{s-1} \\
\end{array}\right]}_{= V}
*
\left[ \begin{array}{r}
b_1 \\
b_2 \\
... \\
b_s \\
\end{array}\right] 
= 
\left[ \begin{array}{r}
1 \\
1/2 \\
... \\
1/s \\
\end{array}\right] 
$$
Falls die Vandermonde-Matrix V invertierbar ist, so lassen sich die Gewichte $b_1, ..., b_s$ bestimmen, sodass die QF $(b_i, c_i)_{i = 1}^{s}$ mindestens Ordnung $s$ hat.
\end{comment}

\begin{definition}
Eine QF heißt symmetrisch, falls für $i = 1,...,s$
\begin{enumerate}
  \item $c_i = 1 - c_{s+1-i}$
  \item $b_i = b_{s+1-i}$
\end{enumerate}
\end{definition}

\begin{example}
MP, TP, Simpson,...
\end{example}

\begin{theorem}
Die maximal erreichbare Ordnung einer symmetrischen QF ist gerade.
\end{theorem}

\begin{theorem}
Sind Knoten $c_1 < c_2 < ... < c_s$ ($c_i \in \mathbb{R}, i = 1,...s$) gegeben, so existieren eindeutig bestimmte Gewichte $b_1 ,..., b_s$ derart, dass die QF $(b_i, c_i)_{i=1}^s$ die maximale Ordnung $p \geq s$ hat. \\
Es gilt $$b_i = \int_0^1 l_i(t) dt$$ mit $$l_i(t) = \frac{\prod_{j=1, j\neq i}^s (t-c_j)}{\prod_{j=1, j\neq i}^s (c_i-c_j)}$$
Bemerkung/Definition
\begin{description}
  \item $l_i$ ist das i-te Lagrangepolynom zu den Knoten $c_i, ...,c_s$. Es gilt $deg(l_i) = s-1$ 
  $$l_i(c_j) = \left\{
\begin{array}{ll}
0 & \,i \neq j \\
1 & \, i = j\\
\end{array}
\right. $$
\end{description}
\end{theorem}
