\subsection{Quadraturfehler}

\underline{Allgemeine Voraussetzung:} $f:[a,b] \rightarrow \mathbb{R}$ sei hinreichend oft differenzierbar ($f$ ist eine glatte Funktion)

\begin{definition}
Der Fehler bei der Approximation des Integrals durch die QF ist 
\begin{align*}
err &= \int_a^b f(x)dx - \sum_{j=0}^{n-1}\left( h_{j+1} \sum_{i=1}^s b_i f(x_j+h_{j+1}c_i)\right)
\intertext{mit $h_{j+1} = x_{j+1}-x_j$}
&= \sum_{j=0}^{n-1} \left( \int_{x_j}^{x_{j+1}} f(x_j + \tau) d\tau - h_{j+1} \sum_{i=1}^s b_i f(x_j + h_{j+1}c_i)\right) &\\
&= \sum_{j=0}^{n-1} h_{j+1} \int_0^1 g_j(\xi) d\xi -h_{j+1} \sum_{i=1}^s b_i g_j(c_i)
\intertext{mit $ g_j(\xi) = f(x_j + \xi h_{j+1})$}
\end{align*}
Der Quadraturfehler auf Teilintervallen $[x_j, x_j+h_{j+1}]$ ist 
\begin{align*}
E(f, x_j, h_{j+1}) &= \int_{x_j}^{x_{j+1}} f(x)dx  - h_{j+1} \sum_{i=1}^s b_i f(x_j + c_i h_{j+1}) &\\
&= h_{j+1} \left( \int_0^1 g_j(\xi) d\xi - \sum_{i=1}^s b_i g_j(c_i) \right)
\end{align*}
\end{definition}

\begin{nothing}[Fehlerabschätzung - 1. Versuch]
Falls $f$ auf $[x_0, x_0+h]$ glatt genug ist und die QF Ordnung $p$ hat, aber nicht Ordnung $p+1$, so erhält man durch Taylorentwicklung um $x_0$ von $f(x_0 + \xi h) = g_0(\xi)$ und $f(x_0+c_ih)$:
\begin{align*}
E(f, x_0, h) &= \sum_{k\geq 0} \frac{h^{k+1}}{k!} \left( \int_0^1 t^k dt - \sum_{i=1}^s b_i c_i^k \right) f^{(k)}(x_0)&\\
&= \frac{h^{p+1}}{p!} \left( \frac{1}{p+1} - \sum_{i=1}^s b_i c_i^p\right) f^{(p)}(x_0) + \underbrace{\mathcal{O}(h^{p+2})}_{Taylorrestglied}
\end{align*}
Die Konstante $C = \frac{1}{p!} \left( \frac{1}{p+1} - \sum_{i=1}^s b_i c_i^p \right)$ heißt Fehlerkonstante. \\
Ist $h$ klein genug, sodass das Taylorrestglied im Vergleich zu $h^{p+1}C f^{(p)}(x_0)$ vernachlässigbar ist, so gilt:
\begin{align*}
err &= \sum_{j=0}^{n-1} E(f, x_j, h)
\intertext{mit $ x_j = x_0+jh$}
&\approx Ch^p \sum_{j=0}^{n-1} hf^{(p)}(x_j)&\\
&\approx Ch^p \int_a^b f^{(p)}(x)dx&\\
&= Ch^p \left(f^{(p-1)}(b) - f^{(p-1)}(a) \right)
\end{align*}
\end{nothing}

\begin{nothing}[Rigorose Fehlerabschätzung]
\begin{description}
  \item
\end{description}
\begin{description}
  \item[Satz 1:]
    Sei $f: [a, b] \rightarrow \mathbb{R}$ $k$-mal stetig differenzierbar ($f \in C^k([a, b])$) und habe die QF Ordnung $p$, so gilt für $h < b-a$ und $k \leq p$\\
    $$E(f, x_0, h) = h^{k+1} \int_0^1 K_k(\tau) f^{(k)}(x_0+\tau k) d\tau,$$
    wobei der Peanokern $K_k(\tau)$ durch 
    $$ K_k(\tau) := \frac{(1-\tau)^k}{k!} - \sum_{i=1}^s b_i \frac{(c_i - \tau)^{k-1}_+}{(k-1)!}, $$
    mit 
    $(\sigma)_+^{k-1} = \left\{
    \begin{array}{ll}
    \sigma ^{k-1} &  \sigma > 0 \\
    0 & \, \textrm{sonst} \\
    \end{array}
    \right. $, gegeben ist.
  \item \begin{proof}[Beweis] 
    Taylorentwicklung mit Integralrestglied und Transformation 
    $$f(x_0 + th) = \sum_{j=0}^{k-1} \frac{(th)^j}{j!} f^{(j)}(x_0) + h^k \int_0^t \frac{(t-\tau)^{k-1}}{(k-1)!} f^{(k)}(x_0+\tau h) d\tau$$
    eingesetzt in (*) und die Verwendung von 
    $$\int_0^{c_i} (c_i - \tau)^{k-1} g(\tau) d\tau = \int_0^1 (c_i-\tau)_+^{k-1} g(\tau) d\tau$$
    liefern
    \begin{align*} 
    E(f, x_0, h) &= h \int_0^1 \left( \sum_{j=0}^{k-1} \frac{(th)^j}{j!} f^{(j)}(x_0) + h^k \int_0^t \frac{(t-\tau)^{k-1}}{(k-1)!} f^{(k)}(x_0+\tau h) d\tau \right)dt &\\
    &\quad- h \sum_{i=1}^s b_i \left( \sum_{j=0}^{k-1} \frac{(c_ih)^j}{j!} f^{(j)}(x_0) + h^k \int_0^{c_i} \frac{(c_i-\tau)^{k-1}}{(k-1)!} f^{(k)}(x_0+c_ih) d\tau \right) &\\
    &\underset{k\leq p}{=} h h^k \left( \int_0^1 \int_0^t \frac{(t-\tau)^{k-1}}{(k-1)!} f^{(k)} (x_0 + \tau h) d\tau dt \right) &\\
    &\quad- h h^k \left(\sum_{i=1}^s \int_0^1 \frac{(c_i-\tau)^{k-1}_+}{(k-1)!} f^{(k)}(x_0+\tau h) d\tau \right)&\\
    &= h h^k \left( \int_0^1 \int_0^1 \frac{(t-\tau)^{k-1}_+}{(k-1)!} f^{(k)}(x_0+\tau h) d\tau dt \right) &\\
    &\quad - h h^k \left(\sum_{i=1}^s b_i \int_0^1 \frac{(c_i - \tau)^{k-1}_+}{(k-1)!} f^{(k)}(x_0+\tau h) d\tau \right)&\\
    &= h^{k+1} \int_0^1 \left( \int_0^1 \frac{(t-\tau)_+^{k-1}}{(k-1)!} dt - \frac{(c_i - \tau)_+^{k-1}}{(k-1)!} \right) f^{(k)}(x_0+\tau h) d\tau &\\
    &= h^{k+1} \int_0^1 K_k(\tau) f^{(k)}(x_0+\tau h) d\tau, &\\
    \end{align*}
    
    da
    \begin{displaymath}
    \int_0^1 \frac{(t-\tau)^{k-1}_+}{(k-1)!} dt
    = \int_0^1 \frac{(t-\tau)^{k-1}}{(k-1)!}
    = \left[ \frac{1}{k!} (t-\tau)^k \right]_{t=\tau}^1
    = \frac{1}{k!} (1-\tau)^k
    \end{displaymath}
    gilt.
    \end{proof}
    
  \item[Satz 2:] (Eigenschaften des Peanokerns) \\
    Für eine QF der Ordnung $p$ gilt für $k \leq p$ ($k, p \in \mathbb{N}$) 
    \begin{enumerate}
      \item $K_k'(\tau) = -K_{k-1}(\tau)$ für $k \geq 2$ und $\tau \neq c_i$ falls $k=2$
      \item $K_k(1) = 0$ für $k \geq 1$, falls $c_i \leq 1$ für $i=1,..., s$
      \item $K_k(0) = 0$ für $k \geq 2$, falls $c_i \leq 1$ für $i=1,..., s$
      \item $\int_0^1 K_p(\tau) = \frac{1}{p!} \left(\frac{1}{p-1} - \sum_{i=1}^s b_i c_i^p \right)=: C$ (Fehlerkonstante $C$ aus (3.2))
      \item $K_1(\tau)$ ist stückweise linear mit Steigung $-1$ und Sprüngen der Höhe $b_i$ an den Stellen $c_i$
    \end{enumerate}
  
  \item
  \begin{proof}[Beweis]
    Eventuell Übungsaufgabe
  \end{proof}
  
  \item[Beispiel:] 
    Mittelpunktsregel: 
      \begin{align*}
      K_1(\tau) &= \frac{(1-\tau)^1}{1!} - 1 \frac{(\frac{1}{2} - \tau)^1_+}{0!} &\\
      &= 1- \tau - \left( \frac{1}{2} - \tau \right)_+^0&\\
      &= \left\{
        \begin{array}{ll}
        1-\tau - 1  & \tau < \frac{1}{2} \\
        1-\tau & \, \tau \geq \frac{1}{2} \\
        \end{array}
      \right.
      \end{align*}
      
      \begin{align*}
      K_2(\tau) &= \frac{(1-\tau)^2}{2!} - 1 \frac{(\frac{1}{2} - \tau)^1_+}{1!}&\\
      &= \frac{1}{2} (1-\tau)^2 - \left( \frac{1}{2} - \tau \right)_+^1&\\
      &= \left\{
        \begin{array}{ll}
        \frac{\tau^2}{2}  & \tau < \frac{1}{2} \\
        \frac{1}{2}(1-\tau)^2 & \, \tau \geq \frac{1}{2} \\
        \end{array}
      \right.
      \end{align*}
    
    \item[Satz 3:] 
      Sei $f \in C^k([a,b])$ und habe die QF $(b_i, c_i)^s_{i=1}$, Ordnung $p \geq k$, so gilt für den Fehler $err$ aus $(3.1)$ 
      $$ \vert err \vert \leq h^k (b-a) \int_0^1 \vert K_k(\tau) \vert d\tau \max_{x \in [a,b]} \vert f^{(k)}(x) \vert, \quad \quad h = \max_{j=1,..,n} h_j.$$
    
    \item \begin{proof}[Beweis]
      Mit Satz $1$ gilt 
      \begin{align*}
      \vert E(f, x_j, h_{j+1}) \vert &\leq h_{j+1}^{k+1} \int_0^1 \vert K_k(\tau) \vert \vert f^{(k)}(x_j+\tau h_{j+1}) \vert d\tau&\\
      &\leq h_{j+1}^{k+1} \int_0^1 \vert K_k(\tau) \vert d\tau \max_{x \in [x_j, x_j + h_{j+1}]} \vert f^{(k)}(x) \vert&\\
      \end{align*}
      Zudem gilt 
      \begin{align*} 
      \vert err \vert &= \vert \sum_{j=0}^{n-1} E(f, x_j, h_{j+1}) \vert &\\
      &\leq \sum_{j=0}^{n-1} \vert E(f, x_j, h_{j+1}) \vert &\\
      &\leq \underbrace{\sum_{j=0}^{n-1} h_{j+1}}_{b-a} \underbrace{h_{j+1}^k}_{\leq h^k} \int_0^1 \vert K_k(\tau) \vert d\tau \underbrace{\max_{x \in [x_j, x_{j+1}]} \vert f^{(k)} (x) \vert}_{\leq \max_{x\in[a,b] \vert f^{(k)}(x) \vert}}
      \end{align*}
      Damit folgt die Behauptung.
    \end{proof}
    
    \item[Beispiele]
      \begin{description}\item\end{description}
      \begin{description}
        \item Für die Mittelpunktsregel (maximale Ordnung = 2) erhält man
      		$$ \vert err \vert \leq h^2 (b-a) \frac{1}{24} \max_{x\in[a,b]} \vert f^{(2)}(x) \vert $$ 
      	\item Für die Trapezregel (maximale Ordnung = 2)
      		$$ \vert err \vert \leq h^2 (b-a) \frac{1}{12} \max_{x\in[a,b]} \vert f^{(2)}(x) \vert $$  
      	\item Für die Simpsonregel (maximale Ordnung = 4)
      		$$ \vert err \vert \leq h^4 (b-a) \frac{1}{2880} \max_{x\in[a,b]} \vert f^{(4)}(x) \vert $$ 
	  \end{description}
      $\rightarrow$ Der Fehler wird klein, falls $h$ klein und die Ordnung $p$ groß wird.
\end{description}
\end{nothing}